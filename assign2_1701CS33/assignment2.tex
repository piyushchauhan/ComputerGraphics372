\documentclass[a4paper, 11pt]{article}
\usepackage{fullpage} % changes the margin
\usepackage{graphicx}
\usepackage{amsmath}
\usepackage{amssymb}
\usepackage{subfig}
\usepackage{hyperref}
\usepackage{listings}
\usepackage{color}
\usepackage{graphics}
\usepackage{caption}

\definecolor{dkgreen}{rgb}{0,0.6,0}
\definecolor{gray}{rgb}{0.5,0.5,0.5}
\definecolor{mauve}{rgb}{0.58,0,0.82}

\lstset{
  frame=tb,
  %language=C++,
  aboveskip=3mm,
  belowskip=3mm,
  showstringspaces=false,
  columns=flexible,
  basicstyle={\small\ttfamily},
  numbers=none,
  numberstyle=\tiny\color{gray},
  keywordstyle=\color{blue},
  commentstyle=\color{dkgreen},
  stringstyle=\color{mauve},
  breaklines=true,
  breakatwhitespace=true,
  tabsize=4
}

\graphicspath{./}

\begin{document}
%Header-Make sure you update this information!!!!
\noindent
{\Huge\textbf{Graphics Assignment - 2}} \\ \\
\Large Arya Das (1701CS11) \hfill 22.01.2020\\

\section*{Introduction}
In this assignment we were taught two variants of a line drawing algorithm. The algorithm is Digital Differential Analyzer or DDA. The two variants taught were Symmetric DDA and Simple DDA.

\section*{Procedure}
DDA is a simple algorithm. We break down the line into large number of steps. We proceed by $\Delta x$ in $x$-axis and $\Delta y$ in $y$-axis and plot the point. We do this until the end of the line is reached.

The main problem in this algorithm is choosing the step size. If we choose a small step size we repeat many pixels and waste time. If we choose a large step size we skip many pixels. The difference between the two variants of DDA is about the step size. In symmetric DDA we divide the differences by 2 until each of them is less than 1. In simple DDA we scale the differences until the largest difference is 1.

\section*{Symmetric DDA}
\subsection*{Code}
\begin{lstlisting}[language=C++]
#include<GL/glut.h>
#include<iostream>

float x1 = -30, y1 = -20, x2 = 120, y2 = 90;
float dx, dy;

void preprocess() {
    float Dx = x2-x1;
    float Dy = y2-y1;

    while(abs(Dx)>1 || abs(Dy)>1) {
        Dx/=2;
        Dy/=2;
    }
    
    dx = Dx;
    dy = Dy;
}

void display() {
    preprocess();

    glClearColor(0.0f, 0.0f, 0.0f, 1.0f);
    glClear(GL_COLOR_BUFFER_BIT);

    glBegin(GL_POINTS);
        glColor3f(1.0f, 0.0f, 0.0f);
        
        float m = x1, n = y1;
        while(abs(x2-m)>5) {
            glVertex2i((GLint)m,(GLint)n);
            m = m+dx;
            n = n+dy;
        }
        
    glEnd();
    glFlush();
}

void reshape(int w, int h) {
    glViewport(0,0,(GLsizei)w,(GLsizei)h);
    glMatrixMode(GL_PROJECTION);
    glLoadIdentity();
    gluOrtho2D(-160,160,-160,160);
    glMatrixMode(GL_MODELVIEW);
}
 
int main(int argc, char** argv) {
   glutInit(&argc, argv);
   glutCreateWindow("Symmetric");
   glutInitWindowSize(320, 320);
   glutInitWindowPosition(50, 50);
   glutDisplayFunc(display);
   glutReshapeFunc(reshape);
   glutMainLoop();
   return 0;
}
\end{lstlisting}

\subsection*{Output}
\begin{center}
\includegraphics[width=0.48\textwidth]{symmetric_dda.png}
\captionof{figure}{Symmetric DDA}
\end{center}

\section*{Simple DDA}
\subsection*{Code}
\begin{lstlisting}[language=C++]
#include<GL/glut.h>
#include<algorithm>
#include<iostream>

float x1 = 30, y1 = -20, x2 = -120, y2 = 90;
float dx, dy;

void preprocess() {
    float Dx = x2-x1;
    float Dy = y2-y1;
    
    dx = Dx/std::max(abs(Dx),abs(Dy));
    dy = Dy/std::max(abs(Dx),abs(Dy));
}

void display() {
    preprocess();

    glClearColor(0.0f, 0.0f, 0.0f, 1.0f);
    glClear(GL_COLOR_BUFFER_BIT);

    glBegin(GL_POINTS);
        glColor3f(1.0f, 0.0f, 0.0f);
        
        float m = x1, n = y1;
        while(abs(x2-m)>5) {
            glVertex2i((GLint)m,(GLint)n);
            m = m+dx;
            n = n+dy;
        }        
    glEnd();

    glFlush();
}

void reshape(int w, int h) {
    glViewport(0,0,(GLsizei)w,(GLsizei)h);
    glMatrixMode(GL_PROJECTION);
    glLoadIdentity();
    gluOrtho2D(-160,160,-160,160);
    glMatrixMode(GL_MODELVIEW);
}
 
int main(int argc, char** argv) {
   glutInit(&argc, argv);
   glutCreateWindow("Simple");
   glutInitWindowSize(320, 320);
   glutInitWindowPosition(50, 50);
   glutDisplayFunc(display);
   glutReshapeFunc(reshape);
   glutMainLoop();
   return 0;
}
\end{lstlisting}

\subsection*{Output}
\begin{center}
\includegraphics[width=0.48\textwidth]{simple_dda.png}
\captionof{figure}{Simple DDA}
\end{center}
\end{document}